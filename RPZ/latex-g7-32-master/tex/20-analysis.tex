\chapter{Аналитический раздел}
\label{cha:analysis}
%
% % В начале раздела  можно напомнить его цель
%
\section{Актуальность проблемы}
С распространением информационных
технологий к интерфейсам программы стали
предъявляться новые требования. Одним из
таких требований стала простота в использо-
вании. Для упрощения взаимодействия, по-
явились диалоговые интерфейсы, позволяю-
щие человеку и ПО взаимодействовать на
языке близком к естественному. В данной
статье рассмотрена система позволяющая ра-
ботать с базой данных.
Программы были призваны упростить
жизнь обычного человека. В конечном итоге
это охватило и процесс работы с данными.
Для хранения и обработки больших объёмов
данных были разработаны базы данных, а для
работы с ними были созданы СУБД (средства
управления базами данных). СУБД зачастую
имели свой собственный язык, создававшийся
из соображений простоты и удобства для
пользователя. В идеале СУБД мог воспользо-
ваться пользователь не обладавший специ-
альным образованием. Однако с увеличением
объёмов данных и их связанности, работать с
базами стало сложнее. Для того чтобы сде-
лать обращение к базе данных ещё проще,
было решено использовать естественный
язык.



\section{Базы данных}

% Обратите внимание, что включается не ../dia/..., а inc/dia/...
% В Makefile есть соответствующее правило для inc/dia/*.pdf, которое
% берет исходные файлы из ../dia в этом случае.
Базы данных можно разделить на два типа. Реляционные и нереляционные, SQL и NoSQL соответственно.
\subsection{SQL базы данных}
SQL базы данных ещё называют реляционными базами данных.\\
Реляционная база данных — это совокупность взаимосвязанных таблиц, каждая из которых содержит информацию об объектах определенного типа. 
Реляционные базы данных организованы на основе отношений  и реалицонной алгебры представляющей собой операции над отношениями.\cite{BD}\\
Реляционная база данных состоит из следующих компонентов:
\begin{center}
\begin{figure}
  \centering
  \includegraphics[width=\textwidth]{./inc/database.png}
  \caption{Основные понятия реляционной БД}
  \label{fig:fig01}
\end{figure}
\end{center}
\cite{RBD}
В реляционной базе данных применяются следующие понятия: 
\begin{itemize}
\item Тип данных -- понятие типа данных в базе данных коррелирует с понятием типа в языках программирования.
\item Домен -- допустимой потенциальное множество значений данного типа.
\item Схема базы данных -- характеризует связи имён атрибутов с именами доменов.
\item Отношение -- характеризует связи атрибутов с данными.
\end{itemize}

\subsection{Графовые базы данных}
Графовые базы данных хранят данные в виде вершин, связей между вершинами и их атрибутов.
Для графовых баз данных применимы следующие понятия:
\begin{itemize}
\item Граф -- абстрактное представление множества объектов, где пары объектов соеденины между собой.
\item Узел -- вершина содержащая некоторые данные.
\item Ребро -- связь между узлами.
\item Свойства -- свойства узла или ребра.
\item Метка -- обозначение типа узла или ребра.
\end{itemize}
Графовая база данных использует интуитивно понятное представление данных, что облегчает задачу проектирования.\\
Гибкая структура базы данных помогает масштабировать базу данных и вносить изменения в её части не затрагивая всю систему.
\subsection{Запросы к базам данных}
Для работы с базой данных применяется некоторое Средство Управления Базой Данных (СУБД), которое может включать в себя некоторый язык для работы с БД.
\subsubsection{SQL}
Для реляционных баз данных применяется язык запросов.SQL (structured query language — «язык структурированных запросов») — декларативный язык программирования, применяемый для создания, модификации и управления данными в произвольной реляционной базе данных, управляемой соответствующей системой управления базами данных (СУБД). SQL основывается на исчислении кортежей.
Рассмотрим некоторые группы операторов SQL.
\begin{itemize}
\item операторы определения данных (Data Definition Language, DDL):
\begin{itemize}
\item CREATE создает объект БД (саму базу, таблицу, представление, пользователя и т. д.),
\item ALTER изменяет объект,
\item DROP удаляет объект;
\end{itemize}
\item операторы манипуляции данными (Data Manipulation Language, DML):
\begin{itemize}
\item SELECT выбирает данные, удовлетворяющие заданным условиям,
\item INSERT добавляет новые данные,
\item UPDATE изменяет существующие данные,
\item DELETE удаляет данные;
\end{itemize}
\end{itemize}
\subsubsection{Cypher}
Cypher это декларативный язык программирования позволяющий работать с графовой базой данных (Neo4j)
Он использует схожие с SQL операторы, но в контексте графовых баз данных.
\begin{lstlisting}[caption={Пример вершины}]
(variable:label {attribute: value,...})
\end{lstlisting}
\begin{itemize}
\item variable - переменная вершины
\item label - метка вершины
\item attribute - атрибут вершины
\item value - значение атрибута
\end{itemize}
\begin{lstlisting}[caption={Пример связи}]
(node)-[con_variable:label {attribute: value,...}]->(node)
\end{lstlisting}
\begin{itemize}
\item con\_variable - переменная связи
\item label - метка связи
\item attribute - атрибут связи
\item value - значение атрибута
\end{itemize}
Оператор Match предоставляет возможность поиска связей и вершин подпадающий под шаблон.
\begin{lstlisting}[caption={Пример запроса}]
MATCH (user)-[:friend]->(follower)
WHERE user.name IN ['Joe', 'John', 'Sara', 'Maria', 'Steve'] AND follower.name =~ 'S.*'
RETURN user.name, follower.name
\end{lstlisting}

\section{Компьютерная Лингвистика}
Для преобразования естественного языка в запрос необходимо извлечь из запроса данные, которые определяют:
\begin{itemize}
\item тип вопроса
\item данные об искомых объектах
\item связь между этими данными
\end{itemize}
Проблемами автоматического извлечения данных из естественного языка занимается компьютерная лингвистика.
Компьютерная лингвистика — раздел науки, изучающий применение математических моделей для описания лингвистических закономерностей. В нашей работе компьютерная лингвистика будет использоваться для получения информации об искомых объектов из текста вопроса. Текст запроса будет подвергнут нескольким видам анализа.
\subsection{Морфологический анализ}
Целью Морфологического анализа является получение данных о форме слова и его основной словоформе.\\
Существует три основных подхода к проведению морфологического анализа. Первый подход часто называют «четкой» морфологией; для русского языка он основан на словаре Зализняка. Второй подход основывается на некоторой системе правил, по заданному слову определяющих его морфологические характеристики; в противоположность первому подходу его называют «нечеткой» морфологией. Третий, вероятностный подход, основан на сочетаемости слов с конкретными морфологическими характеристиками он широко применяется при обработке языков со строго фиксированным порядком слов в предложении и практически неприменим при обработке текстов на русском языке.
\subsection{Синтаксический анализ}
Результатом синтаксического анализа является граф, узлами которого выступают слова предложения; при этом, если два слова связаны каким-либо образом, то соответствующие им вершины графа связаны дугой с определенной окраской. Возможные окраски дуг зависят от языка, на котором написано предложение, а также от выбранного способа представления синтаксической структуры предложения.
Методы синтаксического анализа можно разделить на две группы: методы с фиксированным, заранее заданным набором правил и самообучающиеся методы. Правила представляются не в виде классических продукций («если ..., то ...»), а в виде грамматик, задающих синтаксис языка. Исторически, первым способом описания синтаксиса языка были формальные грамматики. Они задаются в виде четырех компонентов: множество терминальных символов, множество нетерминальных символов, правила вывода и начальный символ. Формальные грамматики хорошо изучены и широко применяются при описании формальных языков (например, языков программирования), но непригодны для описания синтаксиса естественных языков. 
Синтаксический анализ на основе обучающихся систем  заключается в следующем. Разрабатывается множество примеров, содержащих пару — исходное предложение и результат его синтаксического анализа. Этот результат вводится человеком, занимающимся обучением системы, в ответ на каждое подаваемое на вход предложение. Затем, при подаче на вход предложения, не входящего в список примеров, система сама генерирует результат. Для реализации такой обучающейся системы используются такие методы, как нейронные сети, деревья вывода, ILP и методы поиска ближайшего соседа.
\subsection{Семантический анализ}
Семантический анализ текста базируется на результатах синтаксического анализа, получая на входе уже не набор слов, разбитых на предложения, а набор деревьев, отражающих синтаксическую структуру каждого предложения. Поскольку методы синтаксического анализа пока мало изучены, решения целого ряда задач семантической обработки текста базируются на результатах анализа отдельных слов, и вместо синтаксической структуры предложения, анализируются наборы стоящих рядом слов.
Большинство методов семантического анализа, так или иначе, работают со смыслом слов. Следовательно, должна быть какая-то общая для всех методов анализа база, позволяющая выявлять семантические отношения между словами. Такой основой является тезаурус языка. На математическом уровне он представляет собой ориентированный граф, узлами которого являются слова в их основной словоформе. Дуги задают отношения между словами и могут иметь ряд окрасок.\\
\begin{itemize}
\item Синонимия. Слова, связанные дугой с такой окраской, являются синонимами.
\item Антонимия. Слова, связанные дугой с такой окраской, являются антонимами.
\item Гипонимия. Дуги с такой окраской отражают ситуацию, когда одно слово является частным случаем другого (например, слова "мебель" и "стол"). Дуги направлены от общего слова к более частному.
\item Гиперонимия. Отношение, обратное к гипонимии.
\item Экванимия. Дугами с такой окраской связаны слова, являющиеся гипонимами одного и того же слова.
\item Амонимия. Слова, связанные таким отношением, имеют одинаковое написание и произношение, но имеют различный смысл.
\item Паронимия. Данный тип дуги связывает слова, которые часто путают.
\item Конверсивы. Слова, связанные такой окраской, имеют "обратный смысл" (например, "купил" и "продал").
\end{itemize}
\cite{Kling}
Оперируя данными понятиями и взаимосвязью слов в предложении можно извлекать из текста необходимые данные.
\subsection{Семантические сети}
\subsubsection{Понятие семантической сети}
Семантическая сеть — это система знаний, имеющая определенный смысл в виде целостного образа сети,
узлы которой соответствуют понятиям и объектам,
а дуги — отношениям между понятиями и объектами.
Семантическая сеть — способ представления знаний в виде графа, где вершины графа — понятия, а связи между вершинами — некоторые отношения между этими понятиями.\cite{semnet}
Понятием будет являться какой либо объект нашей предметной области.
Отношением же будет являться тип связи его с другими объектами.
Выделяют следующие типы отношений.
\begin{itemize}
\item таксономические («класс – подкласс – экземпляр», «множество – подмножество – элемент» и т.п.). Данный тип отношения называют также отношением AKO (англ. A Kind Of – является разновидностью), IS A (является, это есть) или гипонимии (гипероним – общая сущность; гипоним – частная сущность);

\item  структурные («часть – целое»). Данный тип отношения называют также отношением Part of (является частью), Has part (состоит из, включает в себя), агрегации (лат. aggregatio – присоединение), композиции (лат. compositio – составление, связывание, сложение, соединение) или меронимии (холоним – сущность, включающая в себя другие; мероним – сущность, являющаяся частью другой);

\item родовые («предок» - «потомок»);

\item  производственные («начальник» - «подчиненный»);

\item  функциональные (определяемые обычно глаголами «производит», «влияет» и т.п.);

\item  количественные (больше, меньше, равно и т.п.);

\item  пространственные (далеко от, близко от, за, под, над и т.п.);

\item временные (раньше, позже, в течение и т.п.);

\item  атрибутивные (иметь свойство, иметь значение);

\item  логические (И, ИЛИ, НЕ);

\item  казуальные (причинно-следственные).
\end{itemize}

Отношения можно также классифицировать по степени участия (арности) понятий в отношениях:
\begin{itemize}
\item унарное (рекурсивное) -- отношение связывает понятие само с собой;

\item бинарное -- отношение связывает два понятия;

\item N-арное -- отношение, связывающее более двух понятий.
\end{itemize}
В чём преимущества семантической сети?
Одним из главнейших преимуществ семантической сети для нас здесь является 
близость представления знаний в семантической сети к представлению знаний
используемом человеком, что обеспечивает наглядность такого представления.

\subsubsection{OWL}
Задача построения семантической сети включает следующие подзадачи:
\begin{enumerate}
\item Должны быть выделены объекты предметной области.
\item Между объектами должны быть выделены и реализованы связи.
\end{enumerate}
Для решения этих задач, было решено, в качестве промежуточного этапа использовать Язык Веб Онтологий (OWL).
Основным понятием OWL можно считать класс.
\begin{itemize}
\item Класс -- множество индивидуальных объектов, объединённых по некоторому признаку.
\item Подкласс -- подмножество множества объектов класса.
\item Свойство -- связь между объектами.
\item Подсвойство -- "уточняющее свойство" ("есть родственники", уточняющее "есть 
сестра")
\item Домен -- Множество на котором определено данное свойство
\item Диапазон -- Множество из которого данное свойство выбирает информацию.
\item Индивид -- некоторый индивидуальный объект.
\end{itemize} 
Пользуясь этими понятиями можно с лёгкость описать семантическую сеть.
\section{Запросы на естественном языке}
\subsection{Языки}
В приведённой работе следует учитывать некоторые особенности естественного языка используемого для запроса. Наибольшую роль для нас играет "жёсткость" грамматики в используемом языке, т.к. она влияет на возможность вычленения связей из текста. Чем меньше язык подчинён грамматике, тем больше вероятность допустить ошибку при его парсинге.
\subsection{Виды запросов}
К базе данных могут поступить следующие запросы:
\begin{itemize}
\item Запрос на получение данных по предикату.
\item Запрос на изменение данных по предикату.
\item Запрос на удаление данных по предикату.
\item Запрос на добавление данных.
\end{itemize}
Для реализации этих запросов необходимо в большей или меньшей степени извлекать данные из естественного языка. Чем более жёсткой грамматикой обладает язык, тем качественнее мы сможем это сделать. Однако, даже в правильно полученном дереве могут возникать некоторые коллизии разрешение, которых будет весьма сложной задачей. В частности достаточно сложно понять, что является частью имени какого либо объекта, а что является его свойством.
Всё это требует отдельных ограничений на грамматику естественного языка.

\section{Существующие аналоги}
\subsection{QuePy}
QuePy - фреймворк позволяющий переводить запросы с естественного языка на язык запросов. Проблема данного фреймворка заключается в том, что трансляция происходит с помощью механизма грамматик, что является негибким механизмом.
\section{Выводы}
Проведя анализ предметной области было решено использовать для примера базу данных содержащую информацию о фильмах. Для создания семантической сети будет применена графовая база данных вкупе с онтологией написанной на OWL. Язык будет ограничен только теми предложениями из которых может быть получено синтаксическое дерево, полностью определяющее искомый объект.
Цулью данной работы будет:
получить оценку эффективности данного метода составления запросов, ди-
намику изменения времени обработки запроса взависимости от полученных
свойств.
Для этого надо решить следующие задачи:
\begin{itemize}
\item Написать эмулятор клиента
\item Сравнить запросы по времени
\end{itemize}

% % % % % % % % % % % % % % % % % % % % % % % % % % % % % % % % %