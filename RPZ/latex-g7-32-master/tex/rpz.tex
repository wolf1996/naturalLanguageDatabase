
%% Преамбула TeX-файла

% 1. Стиль и язык
\documentclass[utf8x, 12pt]{G7-32} % Стиль (по умолчанию будет 14pt)

% Остальные стандартные настройки убраны в preamble.inc.tex.
\sloppy

% Настройки стиля ГОСТ 7-32
% Для начала определяем, хотим мы или нет, чтобы рисунки и таблицы нумеровались в пределах раздела, или нам нужна сквозная нумерация.
\EqInChapter % формулы будут нумероваться в пределах раздела
\TableInChapter % таблицы будут нумероваться в пределах раздела
\PicInChapter % рисунки будут нумероваться в пределах раздела

% Добавляем гипертекстовое оглавление в PDF
\usepackage[
bookmarks=true, colorlinks=true, unicode=true,
urlcolor=black,linkcolor=black, anchorcolor=black,
citecolor=black, menucolor=black, filecolor=black,
]{hyperref}

% Изменение начертания шрифта --- после чего выглядит таймсоподобно.
% apt-get install scalable-cyrfonts-tex

\IfFileExists{cyrtimes.sty}
    {
        \usepackage{cyrtimespatched}
    }
    {
        % А если Times нету, то будет CM...
    }

\usepackage{graphicx}   % Пакет для включения рисунков

% С такими оно полями оно работает по-умолчанию:
% \RequirePackage[left=20mm,right=10mm,top=20mm,bottom=20mm,headsep=0pt]{geometry}
% Если вас тошнит от поля в 10мм --- увеличивайте до 20-ти, ну и про переплёт не забывайте:
\geometry{right=20mm}
\geometry{left=30mm}


% Пакет Tikz
\usepackage{tikz}
\usetikzlibrary{arrows,positioning,shadows}

% Произвольная нумерация списков.
\usepackage{enumerate}

% ячейки в несколько строчек
\usepackage{multirow}

% itemize внутри tabular
\usepackage{paralist,array}

% Центрирование подписей к плавающим окружениям
\usepackage[justification=centering]{caption}


% Настройки листингов.
\ifPDFTeX
% 8 Листинги

\usepackage{listings}

% Значения по умолчанию
\lstset{
  basicstyle= \footnotesize,
  breakatwhitespace=true,% разрыв строк только на whitespacce
  breaklines=true,       % переносить длинные строки
%   captionpos=b,          % подписи снизу -- вроде не надо
  inputencoding=koi8-r,
  numbers=left,          % нумерация слева
  numberstyle=\footnotesize,
  showspaces=false,      % показывать пробелы подчеркиваниями -- идиотизм 70-х годов
  showstringspaces=false,
  showtabs=false,        % и табы тоже
  stepnumber=1,
  tabsize=4,              % кому нужны табы по 8 символов?
  frame=single
}

% Стиль для псевдокода: строчки обычно короткие, поэтому размер шрифта побольше
\lstdefinestyle{pseudocode}{
  basicstyle=\small,
  keywordstyle=\color{black}\bfseries\underbar,
  language=Pseudocode,
  numberstyle=\footnotesize,
  commentstyle=\footnotesize\it
}

% Стиль для обычного кода: маленький шрифт
\lstdefinestyle{realcode}{
  basicstyle=\scriptsize,
  numberstyle=\footnotesize
}

% Стиль для коротких кусков обычного кода: средний шрифт
\lstdefinestyle{simplecode}{
  basicstyle=\footnotesize,
  numberstyle=\footnotesize
}

% Стиль для BNF
\lstdefinestyle{grammar}{
  basicstyle=\footnotesize,
  numberstyle=\footnotesize,
  stringstyle=\bfseries\ttfamily,
  language=BNF
}

% Определим свой язык для написания псевдокодов на основе Python
\lstdefinelanguage[]{Pseudocode}[]{Python}{
  morekeywords={each,empty,wait,do},% ключевые слова добавлять сюда
  morecomment=[s]{\{}{\}},% комменты {а-ля Pascal} смотрятся нагляднее
  literate=% а сюда добавлять операторы, которые хотите отображать как мат. символы
    {->}{\ensuremath{$\rightarrow$}~}2%
    {<-}{\ensuremath{$\leftarrow$}~}2%
    {:=}{\ensuremath{$\leftarrow$}~}2%
    {<--}{\ensuremath{$\Longleftarrow$}~}2%
}[keywords,comments]

% Свой язык для задания грамматик в BNF
\lstdefinelanguage[]{BNF}[]{}{
  morekeywords={},
  morecomment=[s]{@}{@},
  morestring=[b]",%
  literate=%
    {->}{\ensuremath{$\rightarrow$}~}2%
    {*}{\ensuremath{$^*$}~}2%
    {+}{\ensuremath{$^+$}~}2%
    {|}{\ensuremath{$|$}~}2%
}[keywords,comments,strings]

% Подписи к листингам на русском языке.
\renewcommand\lstlistingname{\cyr\CYRL\cyri\cyrs\cyrt\cyri\cyrn\cyrg}
\renewcommand\lstlistlistingname{\cyr\CYRL\cyri\cyrs\cyrt\cyri\cyrn\cyrg\cyri}

\else
\usepackage{local-minted}
\fi

% Полезные макросы листингов.
% Любимые команды
\newcommand{\Code}[1]{\textbf{#1}}


\begin{document}

\frontmatter % выключает нумерацию ВСЕГО; здесь начинаются ненумерованные главы: реферат, введение, глоссарий, сокращения и прочее.

% Команды \breakingbeforechapters и \nonbreakingbeforechapters
% управляют разрывом страницы перед главами.
% По-умолчанию страница разрывается.

% \nobreakingbeforechapters
% \breakingbeforechapters

\newcommand{\HRule}{\rule{\linewidth}{.5mm}\\}

\begin{center}

\textsc{\large Московский государственный технический университет имени Н.\,Э.~Баумана}\\[5mm]
\textsc{Факультет <<Информатика и системы управления>>}\\
\textsc{Кафедра <<Программное обеспечение ЭВМ\\и информационные технологии>>}\\[2.5mm]

\includegraphics[scale=.5]{./inc/bmstu-logo.png}\\[2.25cm]

\textsc{\large Расчётно-пояснительная записка}\\
\textsc{к курсовому проекту по базам данных на тему}

\HRule
{\huge \bfseries Организация запросов к базе данных на естественном языке.}
\HRule

\vfill

\begin{flushright}
  \begin{tabular}{lrlc}
    Исполнитель:  &    студент ИУ7-62 & Крестов~С.\,Г.   & \underline{\hspace{3cm}}\\[1cm]
    Руководитель: & преподаватель ИУ7 & Строганов~Ю.\,В. & \underline{\hspace{3cm}}\\[1cm]
  \end{tabular}
\end{flushright}

{\large Москва, \the\year}

\end{center}

\pagenumbering{gobble}
\newpage
\pagenumbering{arabic}


\tableofcontents


\Introduction

Наш век называют эпохой информации, огромное количество знаний хранится в базах данных. Но для работы с ними необходимо владеть средствами работы с базами данных. Конечно же доступ к большинству из них мы имеем через UI(User Interface), предоставляющий полный спектр возможностей, но почему бы не рассмотреть альтернативный способ доступа. В данной работе будет создана тестовая база данных и будет организована(частично) работа с этой базой данных с использованием естественного языка (отчасти ограниченного).

В рамках работы должны быть решены следующие задачи:
\begin{enumerate}
  \item Анализ предметной области;
  \item Показать принципы проектирования базы данных для обработки запросов на естественном языке;
  \item Показать принципы формирования запросов к базе данных на естественном языке.
\end{enumerate}

\mainmatter % это включает нумерацию глав и секций в документе ниже

\chapter{Аналитический раздел}
\label{cha:analysis}
%
% % В начале раздела  можно напомнить его цель
%
\section{Актуальность проблемы}
С распространением информационных
технологий к интерфейсам программы стали
предъявляться новые требования. Одним из
таких требований стала простота в использо-
вании. Для упрощения взаимодействия, по-
явились диалоговые интерфейсы, позволяю-
щие человеку и ПО взаимодействовать на
языке близком к естественному. В данной
статье рассмотрена система позволяющая ра-
ботать с базой данных.
Программы были призваны упростить
жизнь обычного человека. В конечном итоге
это охватило и процесс работы с данными.
Для хранения и обработки больших объёмов
данных были разработаны базы данных, а для
работы с ними были созданы СУБД (средства
управления базами данных). СУБД зачастую
имели свой собственный язык, создававшийся
из соображений простоты и удобства для
пользователя. В идеале СУБД мог воспользо-
ваться пользователь не обладавший специ-
альным образованием. Однако с увеличением
объёмов данных и их связанности, работать с
базами стало сложнее. Для того чтобы сде-
лать обращение к базе данных ещё проще,
было решено использовать естественный
язык.



\section{Базы данных}

% Обратите внимание, что включается не ../dia/..., а inc/dia/...
% В Makefile есть соответствующее правило для inc/dia/*.pdf, которое
% берет исходные файлы из ../dia в этом случае.
Базы данных можно разделить на два типа. Реляционные и нереляционные, SQL и NoSQL соответственно.
\subsection{SQL базы данных}
SQL базы данных ещё называют реляционными базами данных.\\
Реляционная база данных — это совокупность взаимосвязанных таблиц, каждая из которых содержит информацию об объектах определенного типа. 
Реляционные базы данных организованы на основе отношений  и реалицонной алгебры представляющей собой операции над отношениями.\cite{BD}\\
Реляционная база данных состоит из следующих компонентов:
\begin{center}
\begin{figure}
  \centering
  \includegraphics[width=\textwidth]{./inc/database.png}
  \caption{Основные понятия реляционной БД}
  \label{fig:fig01}
\end{figure}
\end{center}
\cite{RBD}
В реляционной базе данных применяются следующие понятия: 
\begin{itemize}
\item Тип данных -- понятие типа данных в базе данных коррелирует с понятием типа в языках программирования.
\item Домен -- допустимой потенциальное множество значений данного типа.
\item Схема базы данных -- характеризует связи имён атрибутов с именами доменов.
\item Отношение -- характеризует связи атрибутов с данными.
\end{itemize}

\subsection{Графовые базы данных}
Графовые базы данных хранят данные в виде вершин, связей между вершинами и их атрибутов.
Для графовых баз данных применимы следующие понятия:
\begin{itemize}
\item Граф -- абстрактное представление множества объектов, где пары объектов соеденины между собой.
\item Узел -- вершина содержащая некоторые данные.
\item Ребро -- связь между узлами.
\item Свойства -- свойства узла или ребра.
\item Метка -- обозначение типа узла или ребра.
\end{itemize}
Графовая база данных использует интуитивно понятное представление данных, что облегчает задачу проектирования.\\
Гибкая структура базы данных помогает масштабировать базу данных и вносить изменения в её части не затрагивая всю систему.
\subsection{Запросы к базам данных}
Для работы с базой данных применяется некоторое Средство Управления Базой Данных (СУБД), которое может включать в себя некоторый язык для работы с БД.
\subsubsection{SQL}
Для реляционных баз данных применяется язык запросов.SQL (structured query language — «язык структурированных запросов») — декларативный язык программирования, применяемый для создания, модификации и управления данными в произвольной реляционной базе данных, управляемой соответствующей системой управления базами данных (СУБД). SQL основывается на исчислении кортежей.
Рассмотрим некоторые группы операторов SQL.
\begin{itemize}
\item операторы определения данных (Data Definition Language, DDL):
\begin{itemize}
\item CREATE создает объект БД (саму базу, таблицу, представление, пользователя и т. д.),
\item ALTER изменяет объект,
\item DROP удаляет объект;
\end{itemize}
\item операторы манипуляции данными (Data Manipulation Language, DML):
\begin{itemize}
\item SELECT выбирает данные, удовлетворяющие заданным условиям,
\item INSERT добавляет новые данные,
\item UPDATE изменяет существующие данные,
\item DELETE удаляет данные;
\end{itemize}
\end{itemize}
\subsubsection{Cypher}
Cypher это декларативный язык программирования позволяющий работать с графовой базой данных (Neo4j)
Он использует схожие с SQL операторы, но в контексте графовых баз данных.
\begin{lstlisting}[caption={Пример вершины}]
(variable:label {attribute: value,...})
\end{lstlisting}
\begin{itemize}
\item variable - переменная вершины
\item label - метка вершины
\item attribute - атрибут вершины
\item value - значение атрибута
\end{itemize}
\begin{lstlisting}[caption={Пример связи}]
(node)-[con_variable:label {attribute: value,...}]->(node)
\end{lstlisting}
\begin{itemize}
\item con\_variable - переменная связи
\item label - метка связи
\item attribute - атрибут связи
\item value - значение атрибута
\end{itemize}
Оператор Match предоставляет возможность поиска связей и вершин подпадающий под шаблон.
\begin{lstlisting}[caption={Пример запроса}]
MATCH (user)-[:friend]->(follower)
WHERE user.name IN ['Joe', 'John', 'Sara', 'Maria', 'Steve'] AND follower.name =~ 'S.*'
RETURN user.name, follower.name
\end{lstlisting}

\section{Компьютерная Лингвистика}
Для преобразования естественного языка в запрос необходимо извлечь из запроса данные, которые определяют:
\begin{itemize}
\item тип вопроса
\item данные об искомых объектах
\item связь между этими данными
\end{itemize}
Проблемами автоматического извлечения данных из естественного языка занимается компьютерная лингвистика.
Компьютерная лингвистика — раздел науки, изучающий применение математических моделей для описания лингвистических закономерностей. В нашей работе компьютерная лингвистика будет использоваться для получения информации об искомых объектов из текста вопроса. Текст запроса будет подвергнут нескольким видам анализа.
\subsection{Морфологический анализ}
Целью Морфологического анализа является получение данных о форме слова и его основной словоформе.\\
Существует три основных подхода к проведению морфологического анализа. Первый подход часто называют «четкой» морфологией; для русского языка он основан на словаре Зализняка. Второй подход основывается на некоторой системе правил, по заданному слову определяющих его морфологические характеристики; в противоположность первому подходу его называют «нечеткой» морфологией. Третий, вероятностный подход, основан на сочетаемости слов с конкретными морфологическими характеристиками он широко применяется при обработке языков со строго фиксированным порядком слов в предложении и практически неприменим при обработке текстов на русском языке.
\subsection{Синтаксический анализ}
Результатом синтаксического анализа является граф, узлами которого выступают слова предложения; при этом, если два слова связаны каким-либо образом, то соответствующие им вершины графа связаны дугой с определенной окраской. Возможные окраски дуг зависят от языка, на котором написано предложение, а также от выбранного способа представления синтаксической структуры предложения.
Методы синтаксического анализа можно разделить на две группы: методы с фиксированным, заранее заданным набором правил и самообучающиеся методы. Правила представляются не в виде классических продукций («если ..., то ...»), а в виде грамматик, задающих синтаксис языка. Исторически, первым способом описания синтаксиса языка были формальные грамматики. Они задаются в виде четырех компонентов: множество терминальных символов, множество нетерминальных символов, правила вывода и начальный символ. Формальные грамматики хорошо изучены и широко применяются при описании формальных языков (например, языков программирования), но непригодны для описания синтаксиса естественных языков. 
Синтаксический анализ на основе обучающихся систем  заключается в следующем. Разрабатывается множество примеров, содержащих пару — исходное предложение и результат его синтаксического анализа. Этот результат вводится человеком, занимающимся обучением системы, в ответ на каждое подаваемое на вход предложение. Затем, при подаче на вход предложения, не входящего в список примеров, система сама генерирует результат. Для реализации такой обучающейся системы используются такие методы, как нейронные сети, деревья вывода, ILP и методы поиска ближайшего соседа.
\subsection{Семантический анализ}
Семантический анализ текста базируется на результатах синтаксического анализа, получая на входе уже не набор слов, разбитых на предложения, а набор деревьев, отражающих синтаксическую структуру каждого предложения. Поскольку методы синтаксического анализа пока мало изучены, решения целого ряда задач семантической обработки текста базируются на результатах анализа отдельных слов, и вместо синтаксической структуры предложения, анализируются наборы стоящих рядом слов.
Большинство методов семантического анализа, так или иначе, работают со смыслом слов. Следовательно, должна быть какая-то общая для всех методов анализа база, позволяющая выявлять семантические отношения между словами. Такой основой является тезаурус языка. На математическом уровне он представляет собой ориентированный граф, узлами которого являются слова в их основной словоформе. Дуги задают отношения между словами и могут иметь ряд окрасок.\\
\begin{itemize}
\item Синонимия. Слова, связанные дугой с такой окраской, являются синонимами.
\item Антонимия. Слова, связанные дугой с такой окраской, являются антонимами.
\item Гипонимия. Дуги с такой окраской отражают ситуацию, когда одно слово является частным случаем другого (например, слова "мебель" и "стол"). Дуги направлены от общего слова к более частному.
\item Гиперонимия. Отношение, обратное к гипонимии.
\item Экванимия. Дугами с такой окраской связаны слова, являющиеся гипонимами одного и того же слова.
\item Амонимия. Слова, связанные таким отношением, имеют одинаковое написание и произношение, но имеют различный смысл.
\item Паронимия. Данный тип дуги связывает слова, которые часто путают.
\item Конверсивы. Слова, связанные такой окраской, имеют "обратный смысл" (например, "купил" и "продал").
\end{itemize}
\cite{Kling}
Оперируя данными понятиями и взаимосвязью слов в предложении можно извлекать из текста необходимые данные.
\subsection{Нивре}
В данном алгоритме к синтактическому дереву предъявляются следующие требования 
\begin{itemize}
\item У дерева должен быть один корень
\item Дерево должно соответствовать определению дерева (отсутствие циклов)
\item Каждая вершина должна быть достижима из корня
\item Дерево должно быть проективно (отношения не должны пересекать друг друа)
\end{itemize}
Состояние алгоритма определяется тремя параметрами $$(S,I,A)$$
где 
\begin{itemize}
\item $S$ - стек
\item $I$ - последовательность необработанных символов
\item $A$ - текущее отношение
\end{itemize}
\begin{center}
\begin{figure}
  \centering
  \includegraphics[width=\textwidth]{./inc/algorithmnivretrans.png}
  \caption{Основные понятия реляционной БД}
  \label{fig:fig01}
\end{figure}
\end{center}
Где,
\begin{itemize}
\item Левая Арка - добавляет арку $n'\rightarrow n$, где $n$ - токен с вершины стека, а $n'$ - следующий токен в последовательности и убирает вершину из стека. Проверяется выполнение лексического соответстви $n$ и $n'$, и отсутствия в списке уже созданных вершин $n'' \rightarrow n$.
\item Правая Арка - добавляет арку $n\rightarrow n'$, где $n$ - токен с вершины стека, а $n'$ - следующий токен в последовательности и убирает вершину из стека.
\item Редуцирование - извлекает вершину из стека. Проверяя наличие головного элемента для данной вершины, для соблюдения проективности.
\item Сдвиг - добавление следующего токена в стек.
\end{itemize}
Такой подход обеспечивает выполнение всех поставленных условий.
\cite{Nivre}
\subsection{Конвингтон}
Алгоритм накладывает на дерево следующие требования.
\begin{itemize}
\item Результат работы - единое дерево
\item Каждое слово имеет только одного предка
\item Проективность
\item Обрабатывается слово за итерацию. Для алгоритма не требуется фраза полностью
\item Алгоритм двигается по списку токенов на один токен
\item Каждая ссылка разрешается по максимально близкому токену
\end{itemize}
На систему накладываются следующие условия.
\begin{itemize}
\item Должна быть реализована грамматика, позволяющая определить возможность связи слов за конечное время
\item В дереве не должно быть коллизий
\item Грамматика не должна определять непроизносимые элементы
\item Атомарность слова - не предполагается никаких действий над структурой слова.
\end{itemize}
Алгоритм предполагает обход слов в предложении, их сопоставление и проверку возможности добавления рассматриваемой связи в дерево.
\cite{Covington}
\subsection{SyntaxNet}
Стохастический метод синтаксического анализа. 
В некоторой степени повторяет алгоритм Нивре.
Одним из отличий алгоритма является тот факт, что правдоподобие отношений свойств проверяется при помощи нейронной сети.
\cite{SyntaxNet}

\subsection{Разбор основанный на правилах}
Алгоритм позволяющий выделять основу слова. Для русского языка, действует простой алгоритм из 4 шагов.
\begin{itemize}
\item Происходит поиск окончаний и их удаление
\item Если слово заканчивается на "и", "и" удаляется
\item Удаление словообразовательных окончаний
\item Удаление окончаний превосходной степени, удвоенных "н" и мягкого знака
\end{itemize}\cite{SnowBall}

\subsection{Разбор основанный на словарях}
Алгоритм получает получить информацию о слове базируясь на обширных словарях. Существует два режима работы алгоритма.
Первый - анализируемое слово представленно в словаре. В этом случае вся необходимая информация берётся из словаря и предоставляется пользователю.\\
Второй - анализируемое слово не находится в словаре. В этом случае возникает два возможных решения.
\begin{itemize}
\item Если слово можно представить как ПРЕФИКС + слово из словаря, то слово разбивается и анализируется слово из словаря.
\item Если предыдущий вариант не дал результатов, используется собранная статистика по окончаниям, на основании которой делается предположение о разборе исходя из последних 5 букв слова.
\end{itemize}\cite{pymorphy}

\subsection{Семантические сети}
\subsubsection{Понятие семантической сети}
Семантическая сеть — это система знаний, имеющая определенный смысл в виде целостного образа сети,
узлы которой соответствуют понятиям и объектам,
а дуги — отношениям между понятиями и объектами.
Семантическая сеть — способ представления знаний в виде графа, где вершины графа — понятия, а связи между вершинами — некоторые отношения между этими понятиями.\cite{semnet}
Понятием будет являться какой либо объект нашей предметной области.
Отношением же будет являться тип связи его с другими объектами.
Выделяют следующие типы отношений.
\begin{itemize}
\item таксономические («класс – подкласс – экземпляр», «множество – подмножество – элемент» и т.п.). Данный тип отношения называют также отношением AKO (англ. A Kind Of – является разновидностью), IS A (является, это есть) или гипонимии (гипероним – общая сущность; гипоним – частная сущность);

\item  структурные («часть – целое»). Данный тип отношения называют также отношением Part of (является частью), Has part (состоит из, включает в себя), агрегации (лат. aggregatio – присоединение), композиции (лат. compositio – составление, связывание, сложение, соединение) или меронимии (холоним – сущность, включающая в себя другие; мероним – сущность, являющаяся частью другой);

\item родовые («предок» - «потомок»);

\item  производственные («начальник» - «подчиненный»);

\item  функциональные (определяемые обычно глаголами «производит», «влияет» и т.п.);

\item  количественные (больше, меньше, равно и т.п.);

\item  пространственные (далеко от, близко от, за, под, над и т.п.);

\item временные (раньше, позже, в течение и т.п.);

\item  атрибутивные (иметь свойство, иметь значение);

\item  логические (И, ИЛИ, НЕ);

\item  казуальные (причинно-следственные).
\end{itemize}

Отношения можно также классифицировать по степени участия (арности) понятий в отношениях:
\begin{itemize}
\item унарное (рекурсивное) -- отношение связывает понятие само с собой;

\item бинарное -- отношение связывает два понятия;

\item N-арное -- отношение, связывающее более двух понятий.
\end{itemize}
В чём преимущества семантической сети?
Одним из главнейших преимуществ семантической сети для нас здесь является 
близость представления знаний в семантической сети к представлению знаний
используемом человеком, что обеспечивает наглядность такого представления.

\subsubsection{OWL}
Задача построения семантической сети включает следующие подзадачи:
\begin{enumerate}
\item Должны быть выделены объекты предметной области.
\item Между объектами должны быть выделены и реализованы связи.
\end{enumerate}
Для решения этих задач, было решено, в качестве промежуточного этапа использовать Язык Веб Онтологий (OWL).
Основным понятием OWL можно считать класс.
\begin{itemize}
\item Класс -- множество индивидуальных объектов, объединённых по некоторому признаку.
\item Подкласс -- подмножество множества объектов класса.
\item Свойство -- связь между объектами.
\item Подсвойство -- "уточняющее свойство" ("есть родственники", уточняющее "есть 
сестра")
\item Домен -- Множество на котором определено данное свойство
\item Диапазон -- Множество из которого данное свойство выбирает информацию.
\item Индивид -- некоторый индивидуальный объект.
\end{itemize} 
Пользуясь этими понятиями можно с лёгкость описать семантическую сеть.
\section{Запросы на естественном языке}
\subsection{Языки}
В приведённой работе следует учитывать некоторые особенности естественного языка используемого для запроса. Наибольшую роль для нас играет "жёсткость" грамматики в используемом языке, т.к. она влияет на возможность вычленения связей из текста. Чем меньше язык подчинён грамматике, тем больше вероятность допустить ошибку при его парсинге.
\subsection{Виды запросов}
К базе данных могут поступить следующие запросы:
\begin{itemize}
\item Запрос на получение данных по предикату.
\item Запрос на изменение данных по предикату.
\item Запрос на удаление данных по предикату.
\item Запрос на добавление данных.
\end{itemize}
Для реализации этих запросов необходимо в большей или меньшей степени извлекать данные из естественного языка. Чем более жёсткой грамматикой обладает язык, тем качественнее мы сможем это сделать. Однако, даже в правильно полученном дереве могут возникать некоторые коллизии разрешение, которых будет весьма сложной задачей. В частности достаточно сложно понять, что является частью имени какого либо объекта, а что является его свойством.
Всё это требует отдельных ограничений на грамматику естественного языка.

\section{Существующие аналоги}
\subsection{QuePy}
QuePy - фреймворк позволяющий переводить запросы с естественного языка на язык запросов.
QuePy использует для запросов английский язык. И переводит его в язык запросов к онтологии. В ходе экспериментов, выяснилось, что система не способна генерировать сложные запросы. \\
Система доступна как из исходного кода, так и с помощью демонстрационной онлайн версии, где система работает с двумя крупными онтологиями.\cite{quepy}
\section{Особенности поставленной задачи}
После анализа предметной области и имеющихся аналогов были выявлены следующие аспекты решаемой задачи. 
\begin{itemize}
\item Качество анализа естественного языка \\
На вход системе подаётся дерево, описывающее запрос. Запрос анализируется при помощи синтаксического анализатора. В случае ошибки семантического аналиатора система получит неверные входные данные, что значительно уменьшит вероятность правильного результата. 
\item Представление данных внутри системы.\\
Для описаных систем также важна структура, которой предстваленны данные в системе. Сущности представления должны соотноситься с сущностями извлекаемыми из запроса. Такой подход позволяет избежать перехода от одних сущностей к другим, что значительно снизит сложность разрабатываемой системы.
\end{itemize}
\section{Выводы}
Проведя анализ предметной области было решено использовать для примера базу данных содержащую информацию о фильмах. Для создания семантической сети будет применена графовая база данных вкупе с онтологией написанной на OWL. Язык будет ограничен только теми предложениями из которых может быть получено синтаксическое дерево, полностью определяющее искомый объект.

% % % % % % % % % % % % % % % % % % % % % % % % % % % % % % % % %
\chapter{Конструкторский раздел}
\label{cha:design}
\section{База данных}
\subsection{Структура базы данных}
В базе данных будут представлены следующие классы сущностей.
\begin{itemize}
\item Films -- класс фильмов
\item Person -- класс людей
\begin{itemize}
\item Actor -- класс актёров
\item Director -- класс режисёров
\end{itemize}
\item Country -- класс страны
\end{itemize}
У каждого класса есть набор связей.
\begin{itemize}
\item Country of birth - связь между человеком и его родиной
\item Involved in - связь между человеком и фильмом, где он участвовал.
\begin{itemize}
\item Acted in - связь между человеком и фильмом где он снимался
\item Produced - связь между режисёром и режисируемым фильмом 
\end{itemize}
\item Involves - обратна Involved in
\begin{itemize}
\item Acted - связь обратная Acted
\item Produced by  - связь обратная Produced
\end{itemize}
\item Release Country - связь между фильмом и выпустившей его страной 
\item Released - связь обратная Release Country
\end{itemize}
Свойства же в OWL определяются как отдельные связи с указанием домена.
\begin{table}[ht]
  \caption{Свойства объектов}
  \begin{tabular}{|c|c|c|}
  \hline
    Имя    & Домен & Определение\\
  \hline
  Link & Thing & Id объекта\\
  \hline
  Birth Year  & Person   & Год рождения человека\\  
  \hline
  Year Of Release & Films & Год выпуска фильма \\
  \hline
  Film Name & Films & Название фильма \\
  \hline
  Person Name & Person & Имя человека \\
  \hline
  Country Name & Country & Название страны \\ 
  \hline
  \end{tabular}
  \label{tab:tabular}
\end{table}
\subsection{Алгоритм заполнения базы данных}
Так как для семантической сети нам необходима некоторая структуризация данных, нам необходимо установить между вершинами дополнительные связи и присвоить их дополнительным классам. Так как из предыдущей части видно, что одна связь может подразумевать более общую связь и обратную связь.
Для восстановления дополнительных связей и классов было решено действовать по следующему алгоритму.
\begin{enumerate}
\item Загрузить данные с сайта
\item Сохранить данные в OWL формате по подготовленной схеме.
\item Выгрузить данные из OWL дополняя все связи и классы согласно схеме OWL.
\end{enumerate}
Таким образом используя OWL мы получаем возможность обобщать и интерпритировать некоторые данные изначально факты. 
\section{Компрьютерная лингвистика}
\subsection{Синтаксическое дерево}
Синтаксическое дерево - граф показывающий зависимость между словами, полученный в ходе синтаксического анализа. 
\begin{figure}
  \centering
  \includegraphics[scale = 0.5]{./inc/SyntTree.png}
  \caption{Синтаксическое дерево}
  \label{fig:fig04}
\end{figure}
Узлами графа являются слова, рёбрами является связь между этими словами.
\subsection{Анализ синтаксического дерева}
В данной работе используется следующий алгоритм извлечения данных. 
\begin{enumerate}
\item Происходит поиск ключевых слов.
\item Если слово найдено, для него начинается извлечение данных (поиск ключевых слов для класса к которому относится обрабатываемое слово).
\item Если в ходе извлечения данных найдено ещё слово,которое может быть связано с исходным, то устанавливается связь.
\end{enumerate}
Данный алгоритм позволяет извлекать данные при наличии явных связей между объектами, однако стоит помнить, что существует множество ситуаций, когда связи в синтаксическом дереве нет. В таких ситуациях следует провести постобработку результатов предыдущего алгоритма или использовать другой алгоритм.
\section{Запросы на естественном языке}
\subsection{Алгоритм обработки запроса на естественном языке}
Обработка запроса на естественном языке будет происходить в несколько этапов. 
\begin{itemize}
\item Извлечение признаков из запроса.
\item Формирование шаблонов.
\item Формулирование шаблонов на языке запросов.
\item Обработка шаблона базой данных.
\end{itemize}
После первого этапа, описанного выше мы получим некоторые классы наших шаблонов, каждый из них будет с определённой точностью описывать требуемый объект и будет содержать список атрибутов со значениями, и информацию о связанных вершинах (для связей). Далее каждой неизвестной присваивается имя под которым она будет фигурировать в запросе. Это имя подставляется в шаблон связи. И это же имя в отдельное строке запроса уточняется при помощи данных хранящихся в объекте. Такая система позволяет быстро создать запрос на языке Cypher и почти очевидна для человека.
\subsection{Классификация запросов}
Из-за используемого на этапе извлечения данных алгоритма мы получаем достаточно жёсткие ограничения на запрос. Запрос классифицируется исходя из следующего критерия -- вида синтаксического дерева. Главным критерием здесь будет являться "обеспеченность связи данными", это значит, что мы знаем имя связи, имеем некоторые признаки или ограничения для вершины-источника и имеем такие же признаки или ограничения для вершины-назначения. Подобные вещи легко показать на синтаксическом дереве, представленном ввиде скобочной структуры.
\begin{lstlisting}
[Tree('Show', [Tree('films', ['me', 'all', Tree('directed', [Tree('Tarantino', ['by', 'Quentin'])])])])]
\end{lstlisting}
Здесь видно что слово 'films' имеет связь с словом 
'directed' оба этих слова являются ключевыми и последнее слово будет учтено, т.к. является именем собственным, в результате запрос будет успешно сформирован.
Однако рассмотрим случай:
\begin{lstlisting}
[Tree('Show', [Tree('films', ['all']), Tree('played', ['where', Tree('ActorsName', ['ActorName'])])])]
\end{lstlisting}
Здесь 'film' и 'played' находятся на одном уровне в дереве и связей между ними нет, и т.к. слово 'show' не является ключевым, мы получим не полностью заданную связь и алгоритм потерпит неудачу в составлении запроса.

\section{План тестирования}
Для тестирования будет составлена выборка из нескольких вопросов и будут проведены тесты на предмет извлечения признаков и поиска сущностей. Также для проверки эффективности алгоритма построения запроса, будет составлено несколько однотипных запросов различной сложности.

%%% Local Variables:
%%% mode: latex
%%% TeX-master: "rpz"
%%% End:

\chapter{Технологический раздел}
\label{cha:impl}
\section{Выбор базы данных}
На этапе выбора базы данных были два варианта.
\subsection{Hypergraph}
HyperGraphDB — это расширяемая, портативная, распределенная, встраиваемая система общего назначения со свободным (open-source) механизмом хранения данных. Эта система разработана специально для проектов использующих возможности искусственного интеллекта и семантического вэба и может использоваться как встраиваемая, объектно-ориентированная база данных для проектов любого масштаба.\cite{HG}
Основным преимуществом Hypergraph была поддержка OWL. Однако, у проекта были проблемы с документацией и не было API для языков кроме java из-за было принято решение отказаться от этого продукта.
\subsection{Neo4j} 
Neo4j — это NoSQL база данных, ориентированная на хранение графов. Преимуществом продукта является декларативный язык запросов Cypher.\\
Помимо этого Neo4j предоставляет API на Python, обширную документацию, программу поддержки начинающих разработчиков, удобный Web интерфейс для свой базы данных и т.д. Однако,Neo4j нативно не поддерживает OWL, однако из-за прочих преимуществ была выбрана именно Neo4j. 
\section{Заполнение Базы данных}
\subsection{OWL IDE}
Для разработки онтологии необходим определённый инструментарий. В нашем случае был выбран Protege.\\
Protégé — это свободный, открытый редактор онтологий и фреймворк для построения баз знаний.
Платформа Protégé поддерживает два основных способа моделирования онтологий посредством редакторов Protégé-Frames и Protégé-OWL. Онтологии, построенные в Protégé, могут быть экспортированы во множество форматов, включая RDF (RDF Schema), OWL и XML Schema.
\subsection{OWL API}
Для работы с нашей онтологией в программе необходимо было использовать OWL Api, однако полноценного API под Python не оказалось, потому пришлось использовать RDFlib, предоставляющий возможности реализованные в RDF, которых по большей части хватило для решения прикладных задач.
\subsection{TMDB}
Было рассмотрено 3 варианта получения данных для заполнения БД.
\begin{enumerate}
\item Вручную.
\item Случайная генерация.
\item Получить из Web.
\end{enumerate}
Первый вариант отпал из-за нерационального использования человекочасов. Второй вариант отпал из-за последующих проблем с проверкой правильности работы программы. Третий вариант делится на два подварианта.\\
\begin{enumerate}
\item Парсинг данных с сайтов.
\item Использование API
\end{enumerate}
После некоторых поисков, был найден API для tmdb под python, и первый вариант отпал из-за излишней сложности. Данная библиотека возвращает данные по запросу уже в формате словаря, что облегчает дальнейшую работу.

\subsection {Neo4j API}
Для работы с базой данных также необходим API. Существует несколько библиотек на Python предоставляющих такой интерфейс, после небольшого изучения было решено выбрать "neo4j-rest-client" так как это хорошодокументированная библиотека, предоставляющая возможность работы как через обёртку так и через запросы на Cypher, что и необходимо для описываемого приложения.
\section{Анализ естественного языка}
\subsection{NLTK}
Для работы с естественным языком был выбран NLTK.
Библиотека NLTK, или NTLK — пакет библиотек и программ для символьной и статистической обработки естественного языка, написанных на языке программирования Python. Содержит графические представления и примеры данных. Сопровождается обширной документацией, включая книгу с объяснением основных концепций, стоящих за теми задачами обработки естественного языка, которые можно выполнять с помощью данного пакета. 

\subsection{Парсинг}
парсинг предложения в NLTK может осуществляться несколькими путями, мы коснёмся трёх
\begin{enumerate}
\item Составление Грамматики
\item Malt Parser
\item Stanford Parser
\end{enumerate}
Первый вариант отбрасывается, т.к. для контекстно-независимых грамматик он почти не подходит и требует много знаний.
\subsubsection{Malt Parser}
Синтаксический парсер написанный на Java. Основывается на машинном обучении, для парсинга использует ряд алгоритмов (Nivre, Convington, Stack,Planar, 2-Planar). Возвращает синтаксическое дерево. Проблема этого парсера заключается в скудной документации. Огромным плюсом этого парсера является наличие предобученных моделей, что позволяет новичкам сразу приступить к анализу текста.
\subsubsection{Stanford Parser}
 Синтаксический парсер написанный на Java. Также основывается на машинном обучении. Но возвращает данные не только в виде дерева, но и в виде графа, более удобного в обработке. Обладает обширной документацией и также предоставляет предобученные выборки на ряд языков. 
 \subsubsection{SyntaxNet}
 Синтаксический анализатор основанный на технологии TensorFlow. Также использует машинное обучение. Нет документации, по-умолчанию есть предобученная модель для русского языка.
%%% Local Variables:
%%% mode: latex
%%% TeX-master: "rpz"
%%% End:

\chapter{Экспериментальный раздел}
\label{cha:research}
В ходе тестов были выделен следующие классы запросов.
\section{Запрос объекта по классу}
Данный класс запросов моделирует ситуацию, когда известен лишь класс искомого объекта.\\
Запрос
\begin{figure}[!h]
\begin{tcolorbox}[colback=white, sharpish corners]
\begin{verbatim}
Фильм .
\end{verbatim}
\end{tcolorbox}
\caption{Запрос}
\end{figure}
В данной ситуации происходит объекта класса "Фильм".
\begin{figure}[!h]
\begin{tcolorbox}[colback=white, sharpish corners]
\begin{verbatim}
['фильм']
\end{verbatim}
\end{tcolorbox}
\caption{Дерево}
\end{figure}
\begin{lstlisting}[caption={Результат}]
MATCH(aaa) - [: is_a * 0..]->(aaaa)
WHERE aaaa.system_name in ['Film']

 return distinct aaa
\end{lstlisting}


\section{Запрос объекта по имени}
Данный запрос описывает ситуацию, когда известно имя объекта, но неизвестен его класс.
Запрос
\begin{figure}[!h]
\begin{tcolorbox}[colback=white, sharpish corners]
\begin{verbatim}
[Piper] .
\end{verbatim}
\end{tcolorbox}
\caption{Запрос}
\end{figure}
В данной ситуации происходит поиск объекта с именем 'Piper'.
\begin{figure}[!h]
\begin{tcolorbox}[colback=white, sharpish corners]
\begin{verbatim}
['[Piper]']
\end{verbatim}
\end{tcolorbox}
\caption{Дерево}
\end{figure}
\begin{lstlisting}[caption={Результат}]
MATCH (aaa) WHERE (aaa.Name='Piper')
 return distinct aaa
\end{lstlisting}


\section{Запрос объекта по свойству и классу}
Данный запрос описывает ситуацию, когда известен класс искомого объекта и некоторое свойство.
Запрос
\begin{figure}[!h]
\begin{tcolorbox}[colback=white, sharpish corners]
\begin{verbatim}
Фильм , снятый  [2001-07-20] .
\end{verbatim}
\end{tcolorbox}
\caption{Запрос}
\end{figure}
В данном запросе описывается следующий объект.
\begin{itemize}
\item класс - фильм
\item свойство - снятый, со значением 2001-07-20
\end{itemize}
\begin{figure}[!h]
\begin{tcolorbox}[colback=white, sharpish corners]
\begin{verbatim}
['фильм', [['снять', [['[2001-07-20]']]]]]
\end{verbatim}
\end{tcolorbox}
\caption{Дерево}
\end{figure}
\begin{lstlisting}[caption={Результат}]
MATCH(aaaa) - [: is_a * 0..]->(aaaaa)
WHERE aaaaa.system_name in ['Film']
MATCH (aaaa) -[aaaaaa*0..1]-> (aaaaaaa) WHERE (aaaa.DateOfRelease in ['2001-07-20'])
 return distinct aaaa
\end{lstlisting}

\section{Запрос объекта по связи и классу}
Данный запрос описывает ситуацию, когда известен класс объекта и имеется какое-либо описание связи, присущей этому объекту и объекта с которым эта связь установлена.
\begin{figure}[!h]
\begin{tcolorbox}[colback=white, sharpish corners]
\begin{verbatim}
Драма где снимался [Elda_Maida] .
\end{verbatim}
\end{tcolorbox}
\caption{Запрос}
\end{figure}
Данный запрос описывает ситуацию. 
\begin{itemize}
\item класс искомого объекта - драма
\item тип связи объекта - снимался 
\item описание связанного объекта - Elda Maida
\end{itemize}
\begin{figure}[!h]
\begin{tcolorbox}[colback=white, sharpish corners]
\begin{verbatim}
['драма', [['сниматься', [['[Elda Maida]']]]]] .
\end{verbatim}
\end{tcolorbox}
\caption{Дерево}
\end{figure}
\begin{lstlisting}[caption={Результат}]
MATCH(aaaa) - [: is_a * 0..]->(aaaaa)
WHERE aaaaa.system_name in ['DramaOntologyClassS7U']
MATCH (aaaa) -[aaaaaa]-> (aaaaaaa) 
WHERE type(aaaaaa) in ['Acted', 'InvolveAsActor']
MATCH (aaaaaaa) WHERE (aaaaaaa.Name='Elda Maida')
 return distinct aaaa
\end{lstlisting}

\section{Простая комбинация простых запросов}
В данной ситуации запрос может состоять из нескольких объектов, каждый из которых описывается простым запросом.\\
\begin{figure}[!h]
\begin{tcolorbox}[colback=white, sharpish corners]
\begin{verbatim}
Актёр снимавшийся в фильме , снятом в [2001-07-20] .
\end{verbatim}
\end{tcolorbox}
\caption{Запрос}
\end{figure}
В данном запросе описываются два объекта
\begin{itemize}
\item Искомый объект описан как класс
\item Вспомогательный объект описан через свойство даты выпуска, со значением "2001-07-20".
\end{itemize}
\begin{figure}[!h]
\begin{tcolorbox}[colback=white, sharpish corners]
\begin{verbatim}
['актёр', [['сниматься', [['фильм', [['снятой', [['[2001-07-20]']]]]]]]]
\end{verbatim}
\end{tcolorbox}
\caption{Дерево}
\end{figure}
\newpage
\begin{lstlisting}[caption={Результат}]
MATCH(aaaaa) - [: is_a * 0..]->(aaaaaa)
WHERE aaaaaa.system_name in ['Actor']
MATCH (aaaaa) -[aaaaaaa]-> (aaaaaaaa) 
WHERE type(aaaaaaa) in ['Acted', 'InvolveAsActor']
MATCH(aaaaaaaa) - [: is_a * 0..]->(aaaaaaaaa)
WHERE aaaaaaaaa.system_name in ['Film']
MATCH (aaaaaaaa) -[aaaaaaaaaa*0..1]-> (aaaaaaaaaaa) WHERE (aaaaaaaa.DateOfRelease in ['2001-07-20'])

 return distinct aaaaa
\end{lstlisting}
Ещё один пример запроса с множеством объектов.
\begin{figure}[!h]
\begin{tcolorbox}[colback=white, sharpish corners]
\begin{verbatim}
Актёр снимавшийся в фильме , срежессированном [Quentin_Tarantino] .
\end{verbatim}
\end{tcolorbox}
\caption{Запрос}
\end{figure}
\begin{figure}[!h]
\begin{tcolorbox}[colback=white, sharpish corners]
\begin{verbatim}
['актёр', [['сниматься', [['фильм', [['срежессировать',
 [['[Quentin Tarantino]']]]]]]]]]
\end{verbatim}
\end{tcolorbox}
\caption{Дерево}
\end{figure}
\begin{lstlisting}[caption={Результат}]
MATCH(aaaaa) - [: is_a * 0..]->(aaaaaa)
WHERE aaaaaa.system_name in ['Actor']
MATCH (aaaaa) -[aaaaaaa]-> (aaaaaaaa) 
WHERE type(aaaaaaa) in ['Acted', 'InvolveAsActor']
MATCH(aaaaaaaa) - [: is_a * 0..]->(aaaaaaaaa)
WHERE aaaaaaaaa.system_name in ['Film']
MATCH (aaaaaaaa) -[aaaaaaaaaa]-> (aaaaaaaaaaa) 
WHERE type(aaaaaaaaaa) in ['Directed', 'InvolveAsDirector']
MATCH (aaaaaaaaaaa) WHERE (aaaaaaaaaaa.Name='Quentin Tarantino')
 return distinct aaaaa
\end{lstlisting}

\subsection{Запрос без указания данных}
Система позволяет построить простой запрос, без указания конкретных данных, что позволяет расширить возможности системы.\\
Пример запроса без указания конкретных данных:
\begin{figure}[!h]
\begin{tcolorbox}[colback=white, sharpish corners]
\begin{verbatim}
Актёр , снявший фильм .
\end{verbatim}
\end{tcolorbox}
\caption{Запрос}
\end{figure}
\begin{figure}[!h]
\begin{tcolorbox}[colback=white, sharpish corners]
\begin{verbatim}
['актёр', [['снять', [['фильм']]]]]
\end{verbatim}
\end{tcolorbox}
\caption{Дерево}
\end{figure}
\newpage
\begin{lstlisting}[caption={Результат}]
MATCH(aaaa) - [: is_a * 0..]->(aaaaa)
WHERE aaaaa.system_name in ['Actor']
MATCH (aaaa) -[aaaaaa*0..1]-> (aaaaaaa) WHERE (aaaa.DateOfRelease in [])OR(type(aaaaaa[0]) in ['Directed'])
 return distinct aaaa
\end{lstlisting}
Этот запрос вернёт, всех кто и снимался в фильме и режессировал фильм (Не обязательно один и тот же). 
Проверочный запрос:\\
\begin{figure}[!h]
\begin{tcolorbox}[colback=white, sharpish corners]
\begin{verbatim}
Режиссёр , снимавшийся в фильме .
\end{verbatim}
\end{tcolorbox}
\caption{Запрос}
\end{figure}
\begin{figure}[!h]
\begin{tcolorbox}[colback=white, sharpish corners]
\begin{verbatim}
['режиссёр', [['сниматься', [['фильм']]]]]
\end{verbatim}
\end{tcolorbox}
\caption{Дерево}
\end{figure}
\begin{lstlisting}[caption={Результат}]
MATCH(aaaa) - [: is_a * 0..]->(aaaaa)
WHERE aaaaa.system_name in ['Director']
MATCH (aaaa) -[aaaaaa]-> (aaaaaaa) 
WHERE type(aaaaaa) in ['Acted', 'InvolveAsActor']
MATCH(aaaaaaa) - [: is_a * 0..]->(aaaaaaaa)
WHERE aaaaaaaa.system_name in ['Film']

 return distinct aaaa
\end{lstlisting}

В этом запросе представлено решение проблемы коллизий.
В 4й строке результата показано, что отношение может быть представлено несколькими типами. Такое произошло потому что под словом "сниматься" в системе зарегистрированы два отношения.
\begin{figure}[!h]
\begin{tcolorbox}[colback=white, sharpish corners]
\begin{verbatim}
{
    "domain": [
        "Actor"
    ],
    "range": [
        "Film"
    ],
    "system_name": "Acted",
    "words": ["играть", "сниматься"]
}
\end{verbatim}
\end{tcolorbox}
\caption{Описание первой связи}
\end{figure}
Приведённые системные имена и слова фигурируют в запросе.
\begin{figure}[!h]
\begin{tcolorbox}[colback=white, sharpish corners]
\begin{verbatim}
{
    "domain": [
        "Film"
    ],
    "range": [
        "Actor"
    ],
    "system_name": "InvolveAsActor",
    "words": ["играть","сниматься"]
}
\end{verbatim}
\end{tcolorbox}
\caption{Описание второй связи}
\end{figure}

\subsection{Запрос к классу}
Система позволяет выполнить запрос к классу. Если заданный запрос описывается каким-либо классом, то система вернёт подходящий класс.\\ 
Пример запроса к типу:\\
Например :
\begin{figure}[!h]
\begin{tcolorbox}[colback=white, sharpish corners]
\begin{verbatim}
Человек , снимавшийся в фильме .
\end{verbatim}
\end{tcolorbox}
\caption{Запрос}
\end{figure}
\begin{figure}[!h]
\begin{tcolorbox}[colback=white, sharpish corners]
\begin{verbatim}
['человек', [['сниматься', [['фильм']]]]]
\end{verbatim}
\end{tcolorbox}
\caption{Дерево}
\end{figure}
\newpage
\begin{lstlisting}[caption={Результат}]
MATCH(aaaa) - [: is_a * 0..]->(aaaaa)
WHERE aaaaa.system_name in ['Person']
MATCH (aaaa) -[aaaaaa]-> (aaaaaaa) 
WHERE type(aaaaaa) in ['Acted', 'InvolveAsActor']
MATCH(aaaaaaa) - [: is_a * 0..]->(aaaaaaaa)
WHERE aaaaaaaa.system_name in ['Film']
 return distinct aaaa
\end{lstlisting}
Такой запрос вернёт всех актёров в системе и сам класс актёра. 
\begin{lstlisting}[caption={Результат выполнения запроса}]
###############################
system_name Actor
###############################
system_name RumiHiiragiOntologyInstanceCT8
DateOfBirthday ['1987-08-01']
Name ['Rumi Hiiragi']
....
\end{lstlisting}

\section{Ошибки в рамках допущений}
\subsection{Ошибки синтаксического анализатора}
Синтаксический анализатор, основанный на стохастическом алгоритме, в некоторых ситуациях даёт некорректный результат, что накладывает на систему требование устойчивости к таким ошибкам.\\
Ошибки синтаксического анализа ведут к следующим  проблемам:
\begin{itemize}
\item нарушение предполагаемой организации семантической сети
\item некорректное соотношение объектов нарушающее логику приложения
\item некорректное соотношение объектов соответствующее логике приложения
\end{itemize}
Все эти ситуации требуют отдельного рассмотрения, поскольку будут влиять на различные аспекты системы.
\begin{itemize}
\item нарушение предполагаемой организации выявляется на этапе перевода сообщения в язык запросов, это самый простой и самый частый тип ошибок синтаксического анализатора.
\item некорректное соотношение объектов нарушающее логику приложения, с точки зрения системы запрос будет абсолютно корректным, выявить без обращения к пользователю невозможно, проявится в виде пустого ответа на запрос пользователя.
\item некорректное соотношение объектов соответствующее логике приложения - как и предыдущий класс ошибок, этот класс ошибок без обращения к пользователю невозможно. В данном случае, есть вероятность непустого ответа системы.
\end{itemize}
Последние два типа ошибок встречаются крайне редко так как система на данном этапе накладывает достаточно жёсткие требования на дерево и получить конкретных примеров не удалось.\\
Пример первого типа ошибок:
\begin{figure}[!h]
\begin{tcolorbox}[colback=white, sharpish corners]
\begin{verbatim}
Человек , рождённый [1952-05-02] , снявшийся в фильме .
\end{verbatim}
\end{tcolorbox}
\caption{Запрос, вызывающий ошибку}
\end{figure}
\begin{figure}[!h]
\begin{tcolorbox}[colback=white, sharpish corners]
\begin{verbatim}
['[1952-05-02]', [['Человек'], ['рождённый', 
[[',']]], ['снявшийся', [[','], ['фильме', [['в']]]]], ['.']]]
\end{verbatim}
\end{tcolorbox}
\caption{Дерево до нормализации}
\end{figure}
Как видно из дерева, в данном запросе нарушено требование, согласно которому, каждый дочерний элемент объекта должен быть связью или свойством. Из-за чего система определила запрос как ошибочный.
\subsection{Ошибки морфологического анализатора}
Ошибки морфологического анализатора сводятся к неверной нормализации слова, что можно исправить при помощи расширения словаря, словом предложенным морфологическим анализатором, как норма.\\
В обычной же ситуации система распознает неправильно переведённое слово как ошибочный токен.
Пример запроса
\begin{figure}[!h]
\begin{tcolorbox}[colback=white, sharpish corners]
\begin{verbatim}
Актёр снимавшийся в фильме , снятом [2001-07-20] .
\end{verbatim}
\end{tcolorbox}
\caption{Запрос, вызывающий ошибку}
\end{figure}
\begin{figure}[!h]
\begin{tcolorbox}[colback=white, sharpish corners]
\begin{verbatim}
['Актёр', [['снимавшийся', [['фильме', [['в'],
 ['снятом', [[','], ['[2001-07-20]']]]]]]], ['.']]]
\end{verbatim}
\end{tcolorbox}
\caption{Дерево после нормализации}
\end{figure}
\begin{figure}[!h]
\begin{tcolorbox}[colback=white, sharpish corners]
\begin{verbatim}
['актёр', [['сниматься', [['фильм', [['снятой',
 [['[2001-07-20]']]]]]]]]]
\end{verbatim}
\end{tcolorbox}
\caption{Дерево после нормализации}
\end{figure}
\\Как видно из приведённых деревьев запроса, токен "снятом" преобразовалось к токену "снятой", что не соответствует действительности. Соответствующее расширение словоря, произошло настройкой следующего описания. 
\begin{figure}[!h]
\begin{tcolorbox}[colback=white, sharpish corners]
\begin{verbatim}
{
    "domain": [
        "Film"
    ],
    "range": [
        "str"
    ],
    "system_name": "DateOfRelease",
    "words": ["выпустить", "снятой", "снять"]
}
\end{verbatim}
\end{tcolorbox}
\caption{Дерево после нормализации}
\end{figure}
\\Как видно из данной картинки, список связанных слов был расширен словом-исключением, что позволило системе корректно отработать данный запрос.
\section{Тестирование производительности}
Проведём проверку эфективности запросов, составленных по вышеописанному алгоритму.
\subsection{Эмуляция клиента}
Процесс эмуляции клиента происходит по принципу увеличения определяющих признаков. 
Для достижения данной цели
\begin{itemize}
\item составляется список признаков объекта
\item выбираются признаки определяющие объект однозначно ($L$) и добавляются в результат
\item выбираются признаки не определяющие однозначно объект
\item инициализируем рабочее множество($W$) списком полученным на предыдущем шаге
\item в цикле до необходимой глубины запроса или опустошения $W$
\begin{itemize}
\item к элементам списка из рабочего множества $W$ попарно добавляем элементы $L$, получаем список $R$
\item выбираем из $R$ список запросов определяющих объект однозначно($K$) и добавляем в результат
\item выбираем из $R$ список запросов не определяющих объект однозначно($\overline{K}$) и добавляем в результат
\item назначаем $\overline{K}$ новым рабочим множеством $W$
\end{itemize}
\end{itemize}
Однако такой подход порождает ряд проблем, связанных с большим количеством запросов к базе данных и сопоставлении пар объект/признак для случаев с запросами использующими несколько объектов.
\subsubsection{Уменьшение количества обрабатываемых запросов}
Проанализировав алгоритм, можно понять, что количество запросов растёт в геометрической прогрессии. Для того, чтобы замедлить рост количества запросов надо уменьшить либо рабочее множество или признаки используемые для его расширения.\\
Для решения этой задачи для каждого запроса из рабочей группы, составляется множество, состоящее из вершин-результатов запроса. Затем, мы сравниваем полученные множества и исключаем из рабочей группы запросы описывающие одни и те же множества вершин. Такой подход приводит к серьёзному уменьшению времени работы программы.
\subsection{Составление сложных запросов}
Основная проблема в составлении запросов с несколькими объектами, заключается в соотнесении объектов и их свойств при расширении запроса.\\
Для этого признаки-расширения были сформированы как полноценные запросы, а связи внутри этих запросов были пронумерованы, что позволяет быстро определить принадлежность свойства объекту, связи объекту и т.д.
\section{Результаты экспериментов}
Результаты эксперимента приведены в таблице.\\
\begin{tabular}{|c|c|}
\hline
Количество признаков в запросе & Время выполнения\\
\hline
1 & 0.0563\\
2 & 0.0578\\ 
3 & 0.0648\\
4 & 0.0798\\
5 & 0.0551\\
6 & 0.0582\\
\hline
\end{tabular}

\subsection{Вывод}
Были выделены классы запросов и рассмотрено поведение системы в случае различных критических ситуаций. Также  был поставлен эксперимент входе которого выяснилось, что изменение количества признаков не влияет принципиально на скорость поиска. Однако, если дополнять запрос, определяющий один объект, то время запроса будет расти. 
%%% Local Variables:
%%% mode: latex
%%% TeX-master: "rpz"
%%% End:


\backmatter %% Здесь заканчивается нумерованная часть документа и начинаются ссылки и
            %% заключение

\Conclusion % заключение к отчёту
В результате проделанной работы мы получили общее понимание проблемы запросов к базе данных на естественном языке. А также мы составили общие принципы обработки таких запросов для графовых баз данных. И протестировали их эффективность по времени.

%%% Local Variables: 
%%% mode: latex
%%% TeX-master: "rpz"
%%% End: 


% % Список литературы при помощи BibTeX
% Юзать так:
%
% pdflatex rpz
% bibtex rpz
% pdflatex rpz

\bibliographystyle{gost780u}
\bibliography{rpz}

%%% Local Variables: 
%%% mode: latex
%%% TeX-master: "rpz"
%%% End: 


\end{document}

%%% Local Variables:
%%% mode: latex
%%% TeX-master: t
%%% End:
